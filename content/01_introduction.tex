%!TEX root = ../main.tex

%- research problem / problem statement
%    - problem definition > define problem you will address (and why) in research
%        - context + background information > who has a problem, when/where does problem arise > what is the cause?
%        - specificity / scope > what you are trying to solve, what will not be tackled?
%        - relevance > why is important for society/profession to solve problem > what happens if not, who will feel consequences?
%    - problem statement > describe specific problem or issue within larger problem that needs / is relevant enough to be solved
%        - should indicate what problem is / where it is occurring
%        - must be single problem, explicitly stated, relevant
%- objective > why you are undertaking research, what you are specifically trying to achieve [6]
%    - goal is not to solve a problem, but should identify what study itself will achieve
%- main research question > what you are trying to determine with your study (derived from problem statement) > results are later used to answer problem statement [5]
%    - => example: [2]
%    - what is the basic question you are trying to answer through your research?
%    - break down main question in sub-questions that together answer main question > what more specific questions do you have to answer to be able to answer main research question?
%    - different research question types (descriptive, comparative etc.)
%- brief description of research design
%	- describe study design (part of research plan) > where, when, how (= research method/-ology > survey, experiment etc.), with whom

\chapter{Introduction}
\label{chp:introduction}

\section{Motivation}
\label{sec:motivation}

Germany's \ac{EEG} is a powerful tool used to accelerate the energy transition and thus, represents a great part of the contributions Germany can make to slow down global climate change. Disappointingly though, one of the major initiatives set forth to achieve these goals is likely to become an obstacle in the future -- namely, the 20-year limitation of the \ac{EEG}'s guaranteed \ac{FIT} for small-scale \acp{PVS} and wind turbines, beginning from the moment of their commissioning. As a result, operators of systems which have become ineligible for governmental aid will have to decide whether to:

\begin{enumerate}[label=(\Alph*)]
  \item continue operations at a throttled rate, hereby withholding renewable capacities
  \item trade excess amounts of energy at public exchanges, resulting in a lower \acs{ROI}~\footnote{A lower \acs{ROI} may be expected because the volumes in excess amounts of energy generated by small-scale installations will have to rival that of energy corporations who already benefit from the economies of scale.}
  \item replace their systems to secure a new guaranteed \ac{FIT}, leading to more e-waste
\end{enumerate}

Neither of these options is satisfying or helpful to mitigate climate change. For exactly this reason, the \ac{EBP} was called to life. It attempts to establish private, autonomous marketplaces for renewable energy at which even small amounts can be traded cheaply and hopefully, more profitably compared to regular energy exchanges~\footnote{Please refer to \cite{stoy2019broker} for an in-depth explanation of the \ac{EBP}. An interactive demo, co-created by the author of this thesis, is available at \url{https://energiebroker.cs.hs-rm.de}}.

To support the ongoing research in this undertaking, a data set shall be created which details to what extent private households equipped with \acsp{PVS}:

\begin{itemize}
  \item feed energy into the grid
  \item obtain energy from the grid
  \item store energy in a home battery (if present)
\end{itemize}

This thesis attempts to provide the means for establishing such a data set across a large number of households as part of an upcoming data collection survey.


\section{Goals and Scope}
\label{sec:goals-and-scope}

This thesis is concerned with the design, implementation and testing of a system that can be used to collect and store the data presented above. It does not cover the process of selecting households, nor does it attempt to estimate the costs for operating such a system or draw any conclusions from the collected data itself. More importantly, the legal framework underpinning the survey is expected to have been clarified in advance. In short, the goals and scope of this thesis are the:

\begin{itemize}
  \item requirements engineering
  \item component design and mapping of requirements thereto
  \item deployment and maintenance plan design
  \item production-grade implementation
  \item field test
\end{itemize}

\section{Thesis Overview}
\label{sec:thesis-overview}
