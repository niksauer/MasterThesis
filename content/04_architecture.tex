%!TEX root = ../main.tex

\chapter{Design \& Implementation}
\label{chp:architecture}

\section{Modeling as Edge Cloud Computing Problem}
\label{sec:modeling-as-edge-cloud-problem}

Driven by applications with strict time-bound requirements such as the \ac{IoT}, \ac{AI} or stream data analytics, a new computing model known as edge computing has emerged in recent years that pushes computations closer to the devices (\textit{edge}) where data is being generated \cite[p.~373]{xiong2018extend} \cite[p.~118]{alam2018orchestration}. This model does not only have the potential to improve application latency and thus, user-experience, but may also strengthen security and minimize bandwidth consumption by not having to transfer all data to centralized infrastructure (\textit{cloud})~\footnote{This centralized infrastructure may either follow an on-premise, off-premise or hybrid approach.} \cite[p.~295]{hoque2017towards}. Edge computing works complementary to the cloud, placing services at the most efficient and logical place between the producers and consumers of data \cite[p.~122]{alam2018orchestration}.

Applied to the problem at hand, one can easily see how a system supporting the data collection survey outlined in \autoref{sec:data-collection-survey} resembles that of a large distributed \ac{IoT} application which leverages edge computing. Multiple geographically dispersed measurement devices gather meter data, analyze it for relevance (\textit{edge computing}), and then transfer the interesting measurements to a centralized data store for further in-depth processing (\textit{cloud computing}~\footnote{In this context, cloud computing does not necessarily imply the elasticity and self-service characteristics typically referred to (comp. \gls{cloud computing} in glossary).}). Therefore, the following sections will discuss the proposed solution's architecture in the context of edge and cloud realms, respectively.


\section{Component Design}
\label{sec:component-design}

Striving for a component-based design represents one of the most important practices in software engineering because it facilitates developers in maintaining a complex system by decomposing it into parts that are easier to conceive, understand and program (comp.~\nameref{sec:monolith-problem} on \autopageref{sec:monolith-problem}). At the same time, the process of dismantling a system should not happen arbitrarily but rather attempt to take the application's domain structure into account to reduce the likelihood of having to remedy large parts afterwards (comp.~\nameref{sec:microservice-decomposition} on \autopageref{sec:microservice-decomposition}).  Accordingly, \autoref{fig:component-design} presents a microservice-based approach (see~\autoref{sec:microservice-definition}) to modeling the system in question. It also gives a first indication as to how the components, i.e. microservices, will interact to achieve the desired behavior.

\begin{figure}[hbt]
  \centering
  \includegraphics[width=\textwidth]{resources/04_architecture/component-design}
  \caption{High-level component design}
  \label{fig:component-design}
\end{figure}

\FloatBarrier


\subsection{Edge}
\label{sec:component-design-edge}

The edge of this system is composed by the set of measurement devices that are to be installed across the various participating households. Each measurement device will be fitted with the following set of microservices:

\begin{description}[format={\storedescriptionlabel}]
  \item[Data Aggregator\label{itm:data-aggregator}]
  \hfill \\
  The \ref{itm:data-aggregator} periodically (every \SI{15}{\minute}) makes measurements for each of the electricity or generation meters connected to this device. Each connection is established on the basis of a dedicated physical measurement probe. Measurements will only be taken if the participant associated with this measurement device has granted his consent. As soon as the participant revokes his consent, measurements will stop indefinitely (comp.~\autoref{sec:survey-consent}). For each measurement made, the \ref{itm:data-aggregator} will extract the set of data points that are of interest to the survey and store those, if not empty, as a new single entry in a database along with the measurement date and identifier of the meter from which it originated. The interaction model with the database can be classified as append-only.

  \textbf{Realizes:} \ref{itm:survey-task-read-meters} on \autopageref{itm:survey-task-read-meters}

  \item[Data Transmitter\label{itm:data-transmitter}]
  \hfill \\
  The \ref{itm:data-transmitter} periodically (every \SI{60}{\minute}, at minimum) transfers all of the measurements stored in the \ref{itm:data-aggregator}'s database to the cloud. Transfers are skipped if the participant associated with this measurement device has revoked or not yet granted his consent. Upon successful transfer, measurements will be deleted to avoid duplicate uploads in the future. Although this sharing of databases violates the database-per-service pattern (comp.~\autoref{sec:database-per-service-pattern}), it is still preferred due to the setup's simplicity. Otherwise, the \ref{itm:data-aggregator} would have to offer a network interface to retrieve and delete or modify measurements. This overhead is not warranted given the fact that the \ref{itm:data-aggregator} exclusively interacts with the database in an append-only mode.

  \textbf{Realizes:} \ref{itm:survey-task-transmit-measurements} on \autopageref{itm:survey-task-transmit-measurements}

  \item[Device Frontend\label{itm:device-frontend}]
  \hfill \\
  The \ref{itm:device-frontend} provides the participant associated with this device with a web-based user interface for granting and revoking his consent, viewing his registration details, as well as the measurements that have been made in and transferred from his household. Additionally, it will list the measurement devices which are in possession by that household and states to which meters and thus, \acsp{PVS}, each device is connected to. All of these details and actions are retrieved from and performed through the cloud. It shall be noted that a (desired) consequence of locating this component on the edge, rather than in the cloud, is that participants will no longer have access to this interface once they return their devices to the research group.

  \textbf{Realizes:} \ref{itm:fr-grant-consent}, \ref{itm:fr-revoke-consent}, \ref{itm:fr-revoke-consent-anonymize}, \ref{itm:fr-view-registration-details}, \ref{itm:fr-view-measurements}
\end{description}


\subsection{Cloud}
\label{sec:component-design-cloud}

The cloud of this system is composed by the following set of microservices:

\begin{description}[format={\storedescriptionlabel}]
  \item[Data \acs{API}\label{itm:data-api}]
  \hfill \\
  The \ref{itm:data-api} acts as the single source of truth for each participant's registration details, consent, measurement devices and the measurements themselves. It offers a network interface to retrieve and modify these details, grant and revoke the consent, and most importantly, add and retrieve measurements to and from a central data store that is accessible to the research group. Further, it takes care of anonymizing a participant's registration details upon revocation of his consent (see~\autoref{sec:survey-anonymization}) and tracks a measurement device's version and last date of activity. It may be argued that this component has a monolithic character due to the breadth of methods offered. Yet, the number of methods is not crucial to this judgement, but rather the degree to which they are related (comp.~\autoref{sec:microservice-definition} and \autoref{sec:microservice-decomposition}).

  \textbf{Enables:} \ref{itm:fr-grant-consent}, \ref{itm:fr-revoke-consent}, \ref{itm:fr-revoke-consent-anonymize}, \ref{itm:fr-view-registration-details}, \ref{itm:fr-view-measurements}, \ref{itm:fr-add-participant}, \ref{itm:fr-modify-registration-details}, \ref{itm:fr-view-consent}, \ref{itm:fr-view-health-status}

  \item[Registration Frontend\label{itm:registration-frontend}]
  \hfill \\
  The \ref{itm:registration-frontend} will allow researchers to add new participants, modify their registration details and check whether the consent of a participant has been granted or revoked. All of these details and actions are retrieved from and performed through the \ref{itm:data-api}.

  \textbf{Realizes:} \ref{itm:fr-add-participant}, \ref{itm:fr-modify-registration-details}, \ref{itm:fr-view-consent}

  \item[Status Frontend\label{itm:status-frontend}]
  \hfill \\
  The \ref{itm:status-frontend} provides administrators with a web-based user interface for viewing the health status of each measurement device. Specifically, it will list details such as the device's version, date of the last measurement and moment of activity. For convenience, it will also state to which household and participant a device belongs and whether the participant has granted or revoked his consent. All of these details are retrieved from the \ref{itm:data-api}.

  \textbf{Realizes:} \ref{itm:fr-view-health-status}
\end{description}


\section{Component Specification}
\label{sec:component-specification}

Having given a high-level overview of the components comprising the system, this section will formalize core aspects of the system's two most important microservices.

\subsection{\ref{itm:data-api}}
\label{sec:component-specification-data-api}

\subsubsection{Data Model}
\label{sec:data-api-data-model}

Given the possibility that one human may wish to act as the contact person of two households which participate in the data collection survey (e.g. his own household and that of his parents), this component will model all data around a logical \texttt{Household} entity rather than that of a \texttt{User}. However, the contact details provided as part of the registration (see~\autoref{sec:survey-registration}) do still semantically belong to an individual user, i.e. the participant. These considerations lead to the initial \ac{ERM} given in \autoref{fig:data-api-erm-user-household}.

\begin{figure}[hbt]
  \centering
  \includegraphics[width=0.9\textwidth]{resources/04_architecture/data-api-erm-1}
  \caption{Modeling of participants and households}
  \label{fig:data-api-erm-user-household}
\end{figure}

\FloatBarrier

Next, the answers to the location and metadata subjects of the registration form may be modeled as entities of the same names as shown in \autoref{fig:data-api-erm-household-metadata-address}.

\begin{figure}[hbt]
  \centering
  \includegraphics[width=0.9\textwidth]{resources/04_architecture/data-api-erm-2}
  \caption{Modeling of a household's metadata and address}
  \label{fig:data-api-erm-household-metadata-address}
\end{figure}

\FloatBarrier

Here, the decision to model the relationship as foreign keys with a unique constraint on the \texttt{Metadata} and \texttt{Address} entities shall be noted. This will lead to a \texttt{0..1} cardinality, meaning that this data does not necessarily have to be supplied for an individual household, even if its presence will be enforced on the application layer\todo{explain intention}. The same result may be achieved through nullable foreign keys on the \texttt{Household} entity, although that approach would necessitate a primary key on the \texttt{Metadata} and \texttt{Address} entities which makes less sense conceptually since these details shall not exist on their own.

Before the remaining registration details are modeled, two types of constants shall be introduced. These constants will be used to differentiate electricity and generation meters, as well as to specify the model number of a meter itself. The latter is needed because the steps to take measurements from a meter will likely vary across models.

\begin{multicols}{2}

\paragraph{\texttt{MeterType}}
\begin{itemize}
  \item \texttt{electricity}
  \item \texttt{pv-generation}
\end{itemize}

\vfill\null
\columnbreak

\paragraph{\texttt{MeterModelNumber}}
\begin{itemize}
  \item \texttt{mt681}
  \item \texttt{mt175}
  \item \texttt{ehz}
  \item \texttt{dd3}
  \item \texttt{easy}
\end{itemize}

\end{multicols}

Based on these constants, \autoref{fig:data-api-erm-household-meter-device-pv} models the relationship between a household's meters and \acsp{PVS}. It also associates each meter with a measurement device to indicate how these objects will be physically connected to each other later. As mentioned in the component design, this \texttt{MeasurementDevice} entity will need to track the device's version and date of last activity. Recalling the fact that microservices are deployed independently (comp.~\autoref{sec:microservice-definition}), the device's version will need to be split across multiple attributes, each tracking a different component running on the edge.

\begin{figure}[hbt]
  \centering
  \includegraphics[width=0.9\textwidth]{resources/04_architecture/data-api-erm-3}
  \caption{Modeling of a household's meters, \acsp{PVS} and measurement devices}
  \label{fig:data-api-erm-household-meter-device-pv}
\end{figure}

\FloatBarrier

Now, only runtime-generated data is missing. Beginning with a participant's consent, \autoref{fig:data-api-erm-household-consent} details how the participation status of a household may be modeled. Again, this consent is associated with the \texttt{Household} entity rather than the \texttt{User} entity because one household may wish to stop sharing data earlier than the other.

\begin{figure}[hbt]
  \centering
  \includegraphics[width=0.9\textwidth]{resources/04_architecture/data-api-erm-4}
  \caption{Modeling of a participant's consent}
  \label{fig:data-api-erm-household-consent}
\end{figure}

\FloatBarrier

Finally, the measurements taken from and transmitted for a particular meter are modeled in \autoref{fig:data-api-erm-meter-measurement}.

\begin{figure}[hbt]
  \centering
  \includegraphics[width=0.9\textwidth]{resources/04_architecture/data-api-erm-5}
  \caption{Modeling of a meter's measurements}
  \label{fig:data-api-erm-meter-measurement}
\end{figure}

\FloatBarrier

Of special note is the \acs{JSON} \texttt{data} attribute on the \texttt{Measurement} entity which will contain the actual data points that have been observed for a meter at a given point in time. In turn, this means that measurements will be stored in a denormalized manner. The shape and amount of data will vary across measurements. Several factors contribute to this modeling decision:

\begin{itemize}
  \item electricity and generation meters inherently produce different data
  \item different meter models may expose different data
  \item the data points of interest may vary across households and regions
\end{itemize}

As a result, the responsibility to ensure a consistent representation of measurement data is pushed to the application layer.


\paragraph{Anonymization Considerations}
In light of the data protection requirements formulated in \autoref{sec:survey-anonymization}, the data model shall be briefly verified for compatibility with the anonymization strategy presented therein.

\ref{itm:anonymization-step-remove-contact} is supported through the nullable foreign \texttt{User} entity key modeled on the \texttt{Household} entity which allows households to exist without participants and thus, without any contact details (comp.~\autoref{fig:data-api-erm-user-household}). On the other hand, \ref{itm:anonymization-step-anonymize-pvs} can be easily realized by modifying the attributes of a particular \acs{PVS} instance. However, because the \texttt{PhotovoltaicSystem} entity tracks the installation date as a \texttt{DATETIME}, this attribute will have to be normalized by setting the date to, for example, the first day of the year, rather than removing the date entirely and only keeping the year (comp.~\autoref{fig:data-api-erm-household-meter-device-pv}).

Lastly, even though \ref{itm:anonymization-step-anonymize-pvs} could be implemented based on the current data model, the following change should be made to further simplify the component's realization. It may be argued that because the process of generating a random location within a larger area is a geospatial operation, i.e. an algorithm which deals with geographic coordinate inputs\todo{add citation}, the component performing such an operation should also maintain geographic coordinates for the models on which it is operating. Currently, however, the \texttt{Household} entity is only associated with an \texttt{Address} entity that resembles a postal address. Therefore, an additional \texttt{Location} entity is introduced in \autoref{fig:data-api-erm-household-location}.

\begin{figure}[hbt]
  \centering
  \includegraphics[width=0.9\textwidth]{resources/04_architecture/data-api-erm-6}
  \caption{Modeling of a household's location}
  \label{fig:data-api-erm-household-location}
\end{figure}

This location should be created simultaneously with the household's address but will then, eventually, outlive it once the anonymization has taken place. To create the initial location, the geographic coordinates which, in this case, are modeled as the latitude and longitude in the \ac{GPS}, may be retrieved based on the household's address (\textit{geocoding})\todo{add citation}.


\subsubsection{Service Interface}
\label{sec:data-api-service-interface}

While the microservice pattern does not dictate any specific kind of network interface for exposing a service's methods (comp.~\autoref{sec:microservice-definition}), \acs{REST} \acsp{API} have become ubiquitous and represent the de-facto industry standard approach \footnote{A full discussion on the principles of \acs{REST} is given in the \autoref{app:theoretical-framework} on \autopageref{sec:rest}.}\todo{add citation}. For that reason, \autoref{tab:data-api-endpoints} breaks down the service's functionalities into a set of \acs{HTTP}-based \acs{REST} endpoints and maps each endpoint to the requirements satisfied thereof. A complete documentation based on the \ac{OAS} format is available as part of this thesis' hand-in.

\begin{table}[hbt]
	\centering
  	\begin{tabularx}{\textwidth}{|c|c|l|X|}
		\hline
		\textbf{\#} & \textbf{Method} & \textbf{Path} & \textbf{Requirements} \\
	    \hline
	    \labeltext[Endpoint~1]{1}{itm:data-endpoint-1} & \texttt{GET} & \texttt{/account} & \ref{itm:fr-view-registration-details} \\
	    \labeltext[Endpoint~2]{2}{itm:data-endpoint-2} & \texttt{GET} & \texttt{/household} & \ref{itm:fr-view-registration-details} \\
	    	\labeltext[Endpoint~3]{3}{itm:data-endpoint-3} & \texttt{POST} & \texttt{/household/consent} & \ref{itm:fr-grant-consent} \\
	    	\labeltext[Endpoint~4]{4}{itm:data-endpoint-4} & \texttt{DELETE} & \texttt{/household/consent} & \ref{itm:fr-revoke-consent}, \ref{itm:fr-revoke-consent-anonymize} \\
	    \labeltext[Endpoint~5]{5}{itm:data-endpoint-5} & \texttt{GET} & \texttt{/household/measurement\_device} & \\
	    	\labeltext[Endpoint~6]{6}{itm:data-endpoint-6} & \texttt{GET} & \texttt{/household/meter} & \ref{itm:fr-view-registration-details} \\
	    \labeltext[Endpoint~7]{7}{itm:data-endpoint-7} & \texttt{GET} & \texttt{/household/meter/:uuid/measurement} & \ref{itm:fr-view-measurements} \\
	    \labeltext[Endpoint~8]{8}{itm:data-endpoint-8} & \texttt{POST} & \texttt{/household/meter/measurement} & \ref{itm:survey-task-read-meters} \\
	    	\labeltext[Endpoint~9]{9}{itm:data-endpoint-9} & \texttt{GET} & \texttt{/household/pv\_system} & \ref{itm:fr-view-registration-details} \\
	    \labeltext[Endpoint~10]{10}{itm:data-endpoint-10} &	\texttt{GET} & \texttt{/management/household} & \ref{itm:fr-view-consent}, \ref{itm:fr-view-health-status} \\
	    	\hline
	\end{tabularx}
  	\caption{Data \acs{API} endpoints mapped to requirements}
  	\label{tab:data-api-endpoints}
\end{table}

\FloatBarrier


\subsubsection{Access Control}
\label{sec:data-api-access-control}

Since this microservice will be publicly exposed for consumption by the edge clients, as well as the \ref{itm:registration-frontend} and \ref{itm:status-frontend} (comp.~\autoref{fig:component-design}), appropriate measures must be put into place to protect the component from unauthorized access. In terms of access control, authentication and authorization are key to these concerns.

\begin{description}

  \item[Authentication\label{itm:data-api-authentication}]
  \hfill \\
  The component will need to implement some form of authentication in order to verify the identify of the user, i.e. the participant, researcher or administrator, for whom access to a service is requested. Verification typically happens on the basis of something that the requesting party should know (e.g. password), posses (e.g. access card) or incarnate (e.g. biometrics)\todo{add citation}.

  \paragraph{Edge clients}
  Edge clients will need to posses a digital certificate and present it with every request. Such certificates are generally used to prove the ownership of a public key by carrying an appropriate signature\todo{add citation}. Moreover, they can be extended with additional metadata that is then also attested by the signature \cite[p.~37]{hummen2013towards}. Based on these cryptographic guarantees, the component will authenticate requests by matching the metadata encoded in a certificate against the stored participant registration details. In particular, it will expect each certificate to encode an email address that belongs to a participant. This lookup scheme is supported by the fact that the \texttt{email} attribute in the \texttt{User} entity has been modeled with a unique constraint (comp.~\autoref{fig:data-api-erm-user-household}), ensuring that any given email address can, at most, identify one participant. Of course, the signature of the presented certificate must also be checked to ensure that it comes from a public key which is exclusively owned by this component or some trusted party associated with the data collection survey. Otherwise, requesting parties could self-attest their identity.

  The main benefit of this authentication method is the ability to preload edge devices with certificates~\footnote{The patent presented in \cite{etchegoyen2013device} describes a method for binding digital certificates to one or more devices. In case of a breach, access for individual devices, rather than users, can be revoked.}~\footnote{\citeauthor{ammar2018internet} warn that embedding a key in a device can represent a security risk, if not regularly rotated, since, eventually, all cryptographic algorithms will be broken \cite[p.~23]{ammar2018internet}.}. In theory, this allows a measurement device to transmit data as soon as it is connected to a meter.

  \paragraph{\ref{itm:registration-frontend} and \ref{itm:status-frontend}}
  These frontends will need to know a shared secret and present it with every request. The secret will collectively identify the research or administrator group, meaning that individual researchers and administrators are not discerned. Such differentiation is not needed as per the requirements presented in \autoref{sec:functional-requirements}.

  \item[Authorization\label{itm:data-api-authorization}]
  \hfill \\
  Once users have been identified and authenticated, the component will need to determine whether they are authorized to use the requested service method. Therefore, \autoref{tab:data-api-endpoints-user-groups} details which user group shall be permitted to which of the service's endpoints. \autoref{tab:data-api-endpoints-permission-scopes} further specifies which set of permissions (\textit{scope}) a user must have to access a particular service endpoint.

\begin{table}[hbt]
	\centering
  	\begin{tabularx}{\textwidth}{|l|X|}
		\hline
		\textbf{User Group} & \textbf{Endpoints} \\
	    \hline
	    Participants & \ref{itm:data-endpoint-1} -- \ref{itm:data-endpoint-9} \\
		Administrators & \ref{itm:data-endpoint-10} \\
	    	\hline
	\end{tabularx}
  	\caption{Data \acs{API} endpoints mapped to user groups}
  	\label{tab:data-api-endpoints-user-groups}
\end{table}

\FloatBarrier

\begin{table}[hbt]
	\centering
  	\begin{tabularx}{\textwidth}{|l|X|}
		\hline
		\textbf{Scope} & \textbf{Endpoints} \\
	    \hline
	    \texttt{account} & \ref{itm:data-endpoint-1} \\
		\texttt{household} & \ref{itm:data-endpoint-2}, \ref{itm:data-endpoint-5}, \ref{itm:data-endpoint-6}, \ref{itm:data-endpoint-9} \\
		\texttt{consent} & \ref{itm:data-endpoint-3}, \ref{itm:data-endpoint-4} \\
		\texttt{measurement-read} & \ref{itm:data-endpoint-7} \\
		\texttt{measurement-write} & \ref{itm:data-endpoint-8} \\
	    	\hline
	\end{tabularx}
  	\caption{\ref{itm:data-api} endpoints mapped to permission scopes}
  	\label{tab:data-api-endpoints-permission-scopes}
\end{table}

\FloatBarrier

	Scopes support the principle of least privilege. For instance, the \ref{itm:data-transmitter} only needs the \texttt{measurement-write} scope to perform its primary task of uploading measurements (in addition to the \texttt{consent} scope to check whether uploads may take place), whereas the \ref{itm:device-frontend} requires the \texttt{measurement-read} scope to retrieve individual measurements for display. These scopes may be granted based on the user's group or other metadata which is encoded in a certificate~\cite[p.~1]{butt2004certificate}.
\FloatBarrier

\end{description}

With regard to shielding data from one participant from that of another, the service interface has already been designed from the perspective of the currently authenticated user (comp.~\ref{itm:data-endpoint-1} -- \ref{itm:data-endpoint-9} in \autoref{tab:data-api-endpoints}). Hence, no additional access control measures are needed.


\subsection{\ref{itm:data-aggregator}}
\label{sec:component-specification-data-aggregator}

\subsubsection{Data Model}
\label{sec:data-aggregator-data-model}

In true microservice fashion, this component will only need to model a single entity which is precisely that of a measurement taken from a meter. Its attributes match those of the \texttt{Measurement} entity introduced in \autoref{fig:data-api-erm-meter-measurement}, except that the \texttt{meter} attribute cannot be modeled as a foreign key. This difference is shown in \autoref{fig:data-aggregator-erm}.

\begin{figure}[hbt]
  \centering
  \includegraphics[width=0.38\textwidth]{resources/04_architecture/data-aggregator-erm}
  \caption{Modeling an edge client's measurements}
  \label{fig:data-aggregator-erm}
\end{figure}

Recalling the fact that the application layer is responsible for ensuring a consistent representation of measurement data, a short explanation shall be given as to how this will be achieved. Since the data collection survey is particularly interested in meters which utilize the \acf{SML} communication protocol to expose internal measurement data (see~\autoref{sec:survey-overview}), its representation of measured values (e.g. counter readings or specific amounts of energy) is decisive for this question. The \ac{SML} protocol uses the \ac{OBIS} to denote data items in metering equipment. Its codes have been standardized in \acs{IEC} 62056-6-1 \cite{iec2017obis}. For this survey, the \ac{OBIS} codes listed in \autoref{tab:relevant-obis-codes} are most relevant\todo{link to all codes}.

\begin{table}[hbt]
	\centering
  	\begin{tabularx}{\textwidth}{|l|l|X|}
		\hline
		\textbf{Hex Code} & \textbf{Short Code} & \textbf{Description} \\
	    \hline
	    \texttt{0100010800ff} & \texttt{1.8.0} & Positive active energy total (in \si{\kilo\watt\hour}) \\
	    	\texttt{0100020800ff} & \texttt{2.8.0} & Negative active energy total (in \si{\kilo\watt\hour}) \\
	    \texttt{0100010700ff} & \texttt{1.7.0} & Positive active instantaneous power (in \si{\kilo\watt}) \\
	    \texttt{0100020700ff} & \texttt{2.7.0} & Negative active instantaneous power (in \si{\kilo\watt}) \\
	    	\texttt{0100100700ff} & \texttt{16.7.0} & Sum active instantaneous power (in \si{\kilo\watt}) \\
	    	\hline
	\end{tabularx}
  	\caption{Relevant \acs{OBIS} codes in metering equipment}
  	\label{tab:relevant-obis-codes}
\end{table}

\FloatBarrier

Given these codes, the component will store the relevant measurement data as a \acs{JSON} array of objects that map each observed \acs{OBIS} code to the value observed for that code. However, because the \acs{OBIS} short code is not uniquely identifiable~\footnote{\ac{OBIS} short codes do not carry the physical medium and channel from which data was generated \cite{promoticObis}.}, the hexadecimal representation shall be used.


\subsubsection{Control Flow}
\label{sec:data-aggregator-control-flow}

Although the component's overall control flow has already been discussed in \autoref{sec:component-design-edge} on \autopageref{itm:data-aggregator}, one crucial aspect has not been mentioned so far. Namely that the electricity and generation meters installed in a household may be factory locked. This means that a meter may not expose all of its internal measurement data via the \ac{SML}-based interface. Therefore, it will be necessary to monitor the input stream for the expected \acs{OBIS} codes (see~\autoref{tab:relevant-obis-codes}), and attempt an unlock if the condition appears to persist over a longer period of time. This revised control flow is depicted in \autoref{fig:data-aggregator-control-flow}.

\begin{figure}[hbt]
  \centering
  \includegraphics[width=\textwidth]{resources/04_architecture/data-aggregator-flowchart}
  \caption{Control flow of \ref{itm:data-aggregator}}
  \label{fig:data-aggregator-control-flow}
\end{figure}

\FloatBarrier

Unlocking a meter involves entering a \acs{PIN} code. This code will, in the most cases, have to be obtained from the household's energy supplier. The actual \acs{PIN} input may then happen physically in advance to this survey or programmatically at runtime through the meter's light-based interface (comp.~\autoref{sec:survey-overview}).


\section{Component Realization}
\label{sec:component-realization}


\section{Deployment Model}
\label{sec:deployment-model}


\section{Maintenance and Support Plan}
\label{sec:maintenance-and-support-plan}
