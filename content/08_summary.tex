%!TEX root = ../main.tex

\chapter{Summary}
\label{chp:summary}

We can all work together to slow down global warming. Equally, we can all work together to collectively gather data which, with luck, will steer us towards actions that can counter this warming. A prime example for this collaborative data collection is the survey presented at the beginning of this thesis. It hopes to establish a data set that details how private households equipped with photovoltaic systems satisfy their energy demands and handle the excess amounts of energy produced. Answers to these questions will help in bringing forth the \acl{EBP} and thus, make sure that photovoltaic systems remain attractive beyond the time frame in which they receive financial governmental aid. Yet, to efficiently gather insights on these topics, a system is needed which can autonomously read and share the data of multiple geographically dispersed electricity and generation meters. This thesis has constructed exactly such a system.

Inspired by the Edge Cloud Computing paradigm used in the \acl{IoT}, this thesis has outlined an architecture that, at its core, decomposes the problem space into a set of measurement devices (\textit{edge}) that will each be connected to one or more meters and which will then regularly transmit observed data to a centralized store (\textit{cloud}). Both of these subsystems have further been divided into a set of microservices which each only deal with a small set of closely related functionalities. The literature review undertaken as part of this thesis has argued that such an approach solves a wide range of problems encountered in traditional monolithic applications. Most notably, the deployment can no longer act as a single point of failure.

With regard to deploying the system, this thesis has extensively discussed the benefits of containerized applications. Using containers as the deployment target and medium of choice has allowed the system to achieve a high degree of portability while also shifting operations away from machine-oriented thinking. Management of these containers is facilitated through a state-of-the-art container orchestration platform~-- namely, Kubernetes -- for which specifically a resource-saving distribution has been chosen, allowing less well-equipped hardware to be used for the measurement devices. 

Next, to cover the needs of maintaining and supporting the system over a long period of time, this thesis has laid out concepts through which components can be remotely upgraded and debugged. Neither of these require any input on the part of the participating households.

During the entire system design process, both functional and non-functional requirements were considered and appropriately called out, resulting in a textual traceability model. At the same time, the completeness of these formal specifications has been verified on the basis of an actual implementation carried out simultaneously. Following the principles of agile software development, products were regularly made available for testing by the project sponsors. This agility is reflected in many of the development techniques employed, including the continuous integration and delivery pipelines set up for this project.

Lastly, to quantitatively asses the quality of the design proposed herein, this thesis has analyzed the measurements collected over a three-week field test. A total of eight measurement devices have captured data from ten meters across eight households and three distinct meter models. The statistical results have suggested a near perfect level of system availability. They have also validated the formal requirements as to how much time must pass between each measurement and how often these measurements should be transmitted to the central data store.

In summary, it can therefore be said that this thesis has put forward a very effective solution for collectively capturing the data necessary to support continued research on the \acl{EBP}. This in turn, will hopefully and eventually slow down global warming. 
