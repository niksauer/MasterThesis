%!TEX root = ../main.tex

\addchap{\langabkverz}
%nur verwendete Akronyme werden letztlich im Abkürzungsverzeichnis des Dokuments angezeigt
%Verwendung:
%		\ac{Abk.}   --> fügt die Abkürzung ein, beim ersten Aufruf wird zusätzlich automatisch die ausgeschriebene Version davor eingefügt bzw. in einer Fußnote (hierfür muss in header.tex \usepackage[printonlyused,footnote]{acronym} stehen) dargestellt
%		\acs{Abk.}   -->  fügt die Abkürzung ein
%		\acf{Abk.}   --> fügt die Abkürzung UND die Erklärung ein
%		\acl{Abk.}   --> fügt nur die Erklärung ein
%		\acp{Abk.}  --> gibt Plural aus (angefügtes 's'); das zusätzliche 'p' funktioniert auch bei obigen Befehlen
%	siehe auch: http://golatex.de/wiki/%5Cacronym
%

% acronym settings
\makeatletter
\newif\if@in@acrolist
\AtBeginEnvironment{acronym}{\@in@acrolisttrue}
\newrobustcmd{\LU}[2]{\if@in@acrolist#1\else#2\fi}

\newcommand{\ACF}[1]{{\@in@acrolisttrue\acf{#1}}}
\makeatother

% start acronyms
\begin{acronym}[HATEOAS ]

%\setlength{\itemsep}{-\parsep}

\acro{EBP}{\acl{EB} Platform}
\acro{EB}{EnergieBroker}
\acro{EEG}{Renewable Energy Sources Act (German: \textit{Erneuerbare-Energien-Gesetz})}
\acro{FIT}{Feed-In Tariff}
\acro{EV}{Electric Vehicle}
\acro{HATEOAS}{Hypermedia as the Engine of Application State}
\acro{PVS}{Photovoltaic System}
\acrodefplural{PVS}[PVSes]{Photovoltaic Systems}
\acro{REST}{Representational State Transfer}
\acro{SOA}{Service-Oriented Architecture}
\acro{CPU}{Central Processing Unit}
\acro{API}{Application Programming Interface}
\acro{IP}{Internet Protocol}
\acro{SRP}{Single Responsibility Principle}
\acro{DDD}{Domain-Driven Design}
\acro{ACID}{Atomicity Consistency Isolation Durability}
\acro{NIST}{National Institute of Standards and Technology}
\acro{VM}{Virtual Machine}
\acro{OS}{Operating System}
\acrodefplural{OS}[OSes]{Operating Systems}
\acro{IaaS}{Infrastructure as a Service}
\acro{PaaS}{Platform as a Service}
\acro{LXC}{Linux Containers}
\acro{MAC}{Mandatory Access Control}
\acro{DoS}{Denial of Service}
\acro{SELinux}{Security-Enhanced Linux}
\acro{NSA}{National Security Agency}
\acro{VCS}{Version Control System}
\acro{OCI}{Open Container Initiative}
\acro{MAPE-K}{Monitor-Analyze-Plan-Execute over a shared Knowledge}
\acro{SLA}{Service-Level Agreement}
\acro{CI}{Continuous Integration}
\acro{CD}{Continuous Deployment}
\acro{CI/CD}{\acl{CI} and \acl{CD}}
\acro{XP}{Extreme Programming}
\acro{ROI}{Return on Investment}
\acro{GDPR}{General Data Protection Regulation}
\acro{MaStR}{Marktstammdatenregister}
\acro{QoS}{Quality of Service}
\acro{IoT}{Internet of Things}
\acro{AI}{Artificial Intelligence}
\acro{HTTP}{Hypertext Transfer Protocol}
\acro{URI}{Uniform Resource Identifier}
\acro{ERM}{Entity-Relationship Model}
\acro{JS}{JavaScript}
\acro{JSON}{\acl{JS} Object Notation}
\acro{GPS}{Global Positioning System}
\acro{OAS}{OpenAPI Specification}
\acro{ITU}{International Telecommunication Union}
\acro{SML}{Smart Message Language}
\acro{OBIS}{Object Identification System}
\acro{IEC}{International Electrotechnical Commission}
\acro{PIN}{Personal Identification Number}
\acro{TLS}{Transport Layer Security}
\acro{mTLS}{Mutual \acs{TLS} Authentication}
\acro{TCP}{Transmission Control Protocol}
\acro{MITM}{Man-in-the-Middle Attack}
\acro{DN}{Distinguished Name}
\acro{CA}{Certificate Authority}
\acrodefplural{CA}[CAs]{Certificate Authorities}
\acro{ITU}{International Telecommunication Union}
\acro{NUTS}{Nomenclature of Territorial Units for Statistics}
\acro{BKG}{Federal Agency for Cartography and Geodesy (German: \textit{Bundesamt für Kartographie und Geodäsie})}
\acro{VDE}{Association for Electrical, Electronic \& Information Technologies (German: \textit{Verband der Elektrotechnik Elektronik Informationstechnik})}
\acro{DIN}{German Institute for Standardization (German: \textit{Deutsche Institut für Normung})}
\acro{ARM}{Advanced RISC Machine}
\acro{CNCF}{Cloud Native Computing Foundation}
\acro{SSH}{Secure Shell Protocol}
\acro{LAN}{Local Area Network}
\acro{ISP}{Internet Service Provider}
\acro{NAT}{Network Address Translation}

\end{acronym}
