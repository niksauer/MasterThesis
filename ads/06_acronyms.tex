%!TEX root = ../main.tex

\addchap{\langabkverz}
%nur verwendete Akronyme werden letztlich im Abkürzungsverzeichnis des Dokuments angezeigt
%Verwendung:
%		\ac{Abk.}   --> fügt die Abkürzung ein, beim ersten Aufruf wird zusätzlich automatisch die ausgeschriebene Version davor eingefügt bzw. in einer Fußnote (hierfür muss in header.tex \usepackage[printonlyused,footnote]{acronym} stehen) dargestellt
%		\acs{Abk.}   -->  fügt die Abkürzung ein
%		\acf{Abk.}   --> fügt die Abkürzung UND die Erklärung ein
%		\acl{Abk.}   --> fügt nur die Erklärung ein
%		\acp{Abk.}  --> gibt Plural aus (angefügtes 's'); das zusätzliche 'p' funktioniert auch bei obigen Befehlen
%	siehe auch: http://golatex.de/wiki/%5Cacronym
%

% acronym settings
\makeatletter
\newif\if@in@acrolist
\AtBeginEnvironment{acronym}{\@in@acrolisttrue}
\newrobustcmd{\LU}[2]{\if@in@acrolist#1\else#2\fi}

\newcommand{\ACF}[1]{{\@in@acrolisttrue\acf{#1}}}
\makeatother

% start acronyms
\begin{acronym}[HATEOAS ]

%\setlength{\itemsep}{-\parsep}

\acro{EBP}{\acl{EB} Platform}
\acro{EB}{EnergieBroker}
\acro{EEG}{Renewable Energy Sources Act (German: \textit{Erneuerbare-Energien-Gesetz})}
\acro{EV}{Electric Vehicle}
\acro{HATEOAS}{Hypermedia as the Engine of Application State}
\acro{PVS}{Photovoltaic System}
\acrodefplural{PVS}[PVSes]{Photovoltaic Systems}
\acro{REST}{Representational State Transfer}
\acro{SOA}{Service-Oriented Architecture}
\acro{CPU}{Central Processing Unit}
\acro{API}{Application Programming Interface}
\acro{IP}{Internet Protocol}
\acro{SRP}{Single Responsibility Principle}
\acro{DDD}{Domain-Driven Design}
\acro{ACID}{Atomicity Consistency Isolation Durability}

\end{acronym}
