%!TEX root = ../main.tex

%
% vorher in Konsole folgendes aufrufen:
%	makeglossaries makeglossaries dokumentation.acn && makeglossaries dokumentation.glo
%

%
% Glossareintraege --> referenz, name, beschreibung
% Aufruf mit \gls{...}
%

\newglossaryentry{scale cube}{
	name={scale cube},
	description={Describes a three-dimensional model for scaling an application. X-axis scaling refers to running multiple instances of an application behind a load balancer. Y-axis scaling splits an application into multiple, distinct services. Z-axis scaling partitions the data to be processed across a set of application instances \cite[p.~36]{messina2016simplified}}
}

\newglossaryentry{cloud computing}{
	name={cloud computing},
	description={\acf{NIST} internationally accepted definition of cloud computing calls for resource pooling where provider’s computing resources are pooled to serve multiple consumers using multi-tenant model with different physical and virtual resources dynamically assigned and reassigned according to consumer demand \cite[p.~81]{bernstein2014containers}}
}

%
%\newglossaryentry{scripting language}{
%	name={scripting language},
%	description={A domain-specific programming language that will only execute in special run-time environments. Examples include: Bash for UNIX, JavaScript for web browsers or Visual Basic for Applications \cite{scriptinglanguage}}
%}
%
%\newglossaryentry{software unit}{
%	name={software unit},
%	description={Piece of software created during the lifecycle of a software system. It is relevant for development, operation and maintenance of that software system. Can be developed, maintained and replaced independently from other units. A unit is atomic in the sense of handling multiple units},
%	plural={software units},
%}
%
%\newglossaryentry{configuration}{
%	name={configuration},
%	description={Set of software units associated to each other that build-up a (part-)system}
%}
%
%\newglossaryentry{configuration management}{
%	name={configuration management},
%	description={Role or organisational unit that exclusively identifies, administrates and offers software units. It controls and documents changes to the unit}
%}
%
%\newglossaryentry{version}{
%	name={version},
%	description={An initial release or re-release of a document, as opposed to a revision resulting from issuing change pages to a previous release \cite{24765-2017}},
%	plural={versions},
%}
%
%\newglossaryentry{variant}{
%	name={variant},
%	description={Variants are created out of a version and exists in parallel. Examples include customising software for specific needs, such as adopting it to different operating systems},
%	plural={variants},
%}
