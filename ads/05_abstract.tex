%!TEX root = ../main.tex

\iflang{de}{%
% Dieser deutsche Teil wird nur angezeigt, wenn die Sprache auf Deutsch eingestellt ist.
\renewcommand{\abstractname}{\langabstract} % Text für Überschrift

% \begin{otherlanguage}{english} % auskommentieren, wenn Abstract auf Deutsch sein soll
\begin{abstract}
%Abstract normalerweise auf Englisch. Siehe:  \url{http://www.dhbw.de/fileadmin/user/public/Dokumente/Portal/Richtlinien_Praxismodule_Studien_und_Bachelorarbeiten_JG2011ff.pdf} (8.3.1 Inhaltsverzeichnis)
%
%Ein "`Abstract"' ist eine prägnante Inhaltsangabe, ein Abriss ohne Interpretation und Wertung einer wissenschaftlichen Arbeit. In DIN 1426 wird das (oder auch der) Abstract als Kurzreferat zur Inhaltsangabe beschrieben.
%
%\begin{description}
%\item[Objektivität] soll sich jeder persönlichen Wertung enthalten
%\item[Kürze] soll so kurz wie möglich sein
%\item[Genauigkeit] soll genau die Inhalte und die Meinung der Originalarbeit wiedergeben
%\end{description}
%
%Üblicherweise müssen wissenschaftliche Artikel einen Abstract enthalten, typischerweise von 100-150 Wörtern, ohne Bilder und Literaturzitate und in einem Absatz.
%
%Quelle: \url{http://de.wikipedia.org/wiki/Abstract} Abgerufen 07.07.2011
%
%Diese etwa einseitige Zusammenfassung soll es dem Leser ermöglichen, Inhalt der Arbeit und Vorgehensweise
%des Autors rasch zu überblicken. Gegenstand des Abstract sind insbesondere
%\begin{itemize}
%\item Problemstellung der Arbeit,
%\item im Rahmen der Arbeit geprüfte Hypothesen bzw. beantwortete Fragen,
%\item der Analyse zugrunde liegende Methode,
%\item wesentliche, im Rahmen der Arbeit gewonnene Erkenntnisse,
%\item Einschränkungen des Gültigkeitsbereichs (der Erkenntnisse) sowie nicht beantwortete Fragen.
%\end{itemize}
%Quelle: \url{http://www.ib.dhbw-mannheim.de/fileadmin/ms/bwl-ib/Downloads_alt/Leitfaden_31.05.pdf}, S.~49
%\end{abstract}
% \end{otherlanguage} % auskommentieren, wenn Abstract auf Deutsch sein soll
}



\iflang{en}{%
% Dieser englische Teil wird nur angezeigt, wenn die Sprache auf Englisch eingestellt ist.
\renewcommand{\abstractname}{\langabstract} % Text für Überschrift

\begin{abstract}
\thispagestyle{plain}
%What is the problem?
%What was done?
%What was discovered?
%What do the findings mean?

The EnergieBroker Platform is a novel idea developed to make small-scale photovoltaic and wind turbine systems attractive even after they have become ineligible to receive a guaranteed feed-in tariff. To support the ongoing research in this undertaking, a data set shall be created which will detail how private households equipped with photovoltaic systems satisfy their energy demands and handle the excess amounts of energy produced. For this purpose, the thesis at hand will guide the reader through the process of designing an end-to-end software-based solution capable of autonomously reading and transmitting data from multiple geographically dispersed electricity and generation meters. Based on the blueprint presented herein, a production-grade implementation is carried out simultaneously and tested against a real set of households. Statistical analysis has shown that this system is highly reliable, while also validating conformance with the core functional requirements. 

\end{abstract}

%\renewcommand{\abstractname}{Zusammenfassung}
%
%\begin{abstract}
%\thispagestyle{empty}
%Die EnergieBroker Plattform ist ein Projekt, das ins Leben gerufen wurde, um kleine Photovoltaik- und Windkraftanlagen auch dann noch attraktiv zu machen, wenn sie keinen Anspruch mehr auf eine Einspeisevergütung haben. Zur Unterstützung der laufenden Forschung in diesem Vorhaben soll ein Datensatz erstellt werden, der zeigt, wie private Haushalte mit Photovoltaikanlagen ihren Energiebedarf decken und mit dem überschüssig produzierten Strom umgehen. Zu diesem Zweck führt die vorliegende Arbeit den Leser durch den Prozess der Entwicklung einer softwarebasierten Lösung, die in der Lage ist, Daten von mehreren geografisch verteilten Strom- und Erzeugungszählern zu erheben. Auf Grundlage der vorgestellten Architektur wird eine Implementierung durchgeführt und in echten Haushalten getestet. Die statistische Auswertung hat gezeigt, dass dieses System sehr zuverlässig ist und im Einklang mit den wichtigsten funktionalen Anforderungen arbeitet. 
%\end{abstract}	
}
