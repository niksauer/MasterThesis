%!TEX root = ../main.tex

\iflang{de}{%
% Dieser deutsche Teil wird nur angezeigt, wenn die Sprache auf Deutsch eingestellt ist.
\renewcommand{\abstractname}{\langabstract} % Text für Überschrift

% \begin{otherlanguage}{english} % auskommentieren, wenn Abstract auf Deutsch sein soll
\begin{abstract}
%Abstract normalerweise auf Englisch. Siehe:  \url{http://www.dhbw.de/fileadmin/user/public/Dokumente/Portal/Richtlinien_Praxismodule_Studien_und_Bachelorarbeiten_JG2011ff.pdf} (8.3.1 Inhaltsverzeichnis)
%
%Ein "`Abstract"' ist eine prägnante Inhaltsangabe, ein Abriss ohne Interpretation und Wertung einer wissenschaftlichen Arbeit. In DIN 1426 wird das (oder auch der) Abstract als Kurzreferat zur Inhaltsangabe beschrieben.
%
%\begin{description}
%\item[Objektivität] soll sich jeder persönlichen Wertung enthalten
%\item[Kürze] soll so kurz wie möglich sein
%\item[Genauigkeit] soll genau die Inhalte und die Meinung der Originalarbeit wiedergeben
%\end{description}
%
%Üblicherweise müssen wissenschaftliche Artikel einen Abstract enthalten, typischerweise von 100-150 Wörtern, ohne Bilder und Literaturzitate und in einem Absatz.
%
%Quelle: \url{http://de.wikipedia.org/wiki/Abstract} Abgerufen 07.07.2011
%
%Diese etwa einseitige Zusammenfassung soll es dem Leser ermöglichen, Inhalt der Arbeit und Vorgehensweise
%des Autors rasch zu überblicken. Gegenstand des Abstract sind insbesondere
%\begin{itemize}
%\item Problemstellung der Arbeit,
%\item im Rahmen der Arbeit geprüfte Hypothesen bzw. beantwortete Fragen,
%\item der Analyse zugrunde liegende Methode,
%\item wesentliche, im Rahmen der Arbeit gewonnene Erkenntnisse,
%\item Einschränkungen des Gültigkeitsbereichs (der Erkenntnisse) sowie nicht beantwortete Fragen.
%\end{itemize}
%Quelle: \url{http://www.ib.dhbw-mannheim.de/fileadmin/ms/bwl-ib/Downloads_alt/Leitfaden_31.05.pdf}, S.~49
%\end{abstract}
% \end{otherlanguage} % auskommentieren, wenn Abstract auf Deutsch sein soll
}



\iflang{en}{%
% Dieser englische Teil wird nur angezeigt, wenn die Sprache auf Englisch eingestellt ist.
\renewcommand{\abstractname}{\langabstract} % Text für Überschrift

\begin{abstract}
\thispagestyle{plain}
%What is the problem?
%What was done?
%What was discovered?
%What do the findings mean?

\end{abstract}
}
